% introduction (1-2 pages)
\chapter*{Увод}
\addcontentsline{toc}{chapter}{Увод}

% programming languages
% Компютрите в основата си са прости машини, които четат инструкции и ги изпълняват една след друга. Инструкциите, които един компютър разбира, са много прости и за да се направи нещо по-сложно е нужен голям брой от тях. Това прави разработката на софтуер изцяло на асемблер бавен, сложен и труден процес. Именно за това са били създадени първите езици от високо ниво ({\tt COBOL}, {\tt Fortran}, {\tt C} и др.), които правят писането на код по-лесно, като автоматизират частично или напълно части от процеса на програмиране и правят кода по-четим. Цената на това удобство е загуба на контрол върху хардуера и по-бавен код. Колкото по-високо е нивото на езика, толкова повече се засилва този ефект.

% operating systems
За да се изпълняват няколко програми на един компютър е нужна още една програма, която да ги координира (scheduler). При настолните компютри тази функция я изпълнява операционната система.
Операционна система е програма, която се грижи всички останали програми да не си пречат (пример за това би било две програми да записват данни в едно и също парче памет) и ако има повече програми, отколкото ядра на процесора, операционната система се грижи да разпределя времето на процесора между програмите. Това става като се отнеме пълният контрол на приложенията върху хардуера и им се даде опростен интерфейс, който минава през операционната система.

% bare metal
Не винаги е добра идея да се използва операционна система. Приложение, което не ползва операционна система, се нарича bare-metal ("гол-метал") приложение. Това са най-вече приложения, които вървят върху маломощен хардуер (микроконтролер) или се нуждаят от максимално бързодействие. Поради липсата на операционна система тези приложения трябва сами да извършват някои от нейните функции.

% QEMU & Virtio
Хипервайзор е програма, която позволява изпълнението на няколко операционни системи или bare-metal приложения върху едно устройство едновременно. За разлика от операционна система то не скрива хардуерът, а само го разделя между отделните приложения чрез специални хардуерни инструкции. По този начин не се губи почти никакво бързодействие при изчисления (понеже хипервайзорът не се намесва), но се губи бързодействие при комуникация с външния свят (понеже хипервайзорът трябва да провери дали комуникацията е позволена).
Хипервайзорите се делят на два типа. % TODO

% x86
Процесорна архитектура е колекция от правила, които всеки процесор, имплементиращ дадената архитектура, трябва да следва. Архитектурата дефинира списъка от инструкции, които процесора трябва или може да изпълнява и какво трябва да прави всяка една от тях. Това улеснява разработката на софтуер и подмяната на хардуер, защото софтуер, написан за един процесор, ще работи и на друг процесор със същата архитектура.

% encryption
% цел на проекта
След разглеждане на горепосочените описания, целта на проекта става ясна: приложението трябва да работи във виртуална машина, създадена от QEMU, без да използва операционна система. Това, което ще прави приложението, е да криптира и декриптира данни с асиметрично криптиране, получени от някакъв вход. Тази апликация има две причини да е подходяща за bare-metal:
\begin{enumerate}
  \item Скорост. Асиметричното криптиране е сравнително бавно, което го прави неподходящо за доста цели. Премахването на операционната система ще забърза поне малко имплементацията.
  \item Сигурност. Ако ключът за декриптиране се генерира и складира от едно единствено устройство, то устройството един вид се превръща във физически ключ. Понеже устройството няма операционна система, то няма как да бъде заразено с вирус и единствено може да бъде хакнато чрез дупка в сигурността на интерфейса или чрез физически достъп.
\end{enumerate}

% FIXME - this is rubbish and it doesn't flow naturally

