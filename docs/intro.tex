% introduction (1-2 pages)
\chapter*{Увод}
\addcontentsline{toc}{chapter}{Увод}

% bare-metal

% Компютрите в основата си са прости машини, които четат инструкции и ги изпълняват една след друга. Инструкциите, които един компютър разбира, са много прости и за да се направи нещо по-сложно е нужен голям брой от тях. Това прави разработката на софтуер изцяло на асемблер бавен, сложен и труден процес. Именно за това са били създадени първите езици от високо ниво ({\tt COBOL}, {\tt Fortran}, {\tt C} и др.), които правят писането на код по-лесно, като автоматизират частично или напълно части от процеса на програмиране и правят кода по-четим. Цената на това удобство е загуба на контрол върху хардуера и по-бавен код. Колкото по-високо е нивото на езика, толкова повече се засилва този ефект.

За да се изпълняват няколко програми на един компютър е нужна операционна система. Операционната система е програма, която се грижи всички останали програми да не си пречат (пример за това би било две програми да записват данни в едно и също парче памет) и ако има повече програми, отколкото ядра на процесора, операционната система се грижи да разпределя времето на процесора между програмите. Това става като се отнеме пълният контрол на приложенията върху хардуера и им се даде опростен интерфейс, който минава през операционната система.
Не винаги е добра идея да се използва операционна система. Приложение, което не ползва операционна система, се нарича bare-metal ("гол-метал") приложение. Това са най-вече приложения, които вървят върху маломощен хардуер (микроконтролер) или се нуждаят от максимално бързодействие. Поради липсата на операционна система тези приложения трябва сами да извършват някои от нейните функции.

% x86
Процесорна архитектура е колекция от правила, които всеки процесор, имплементиращ дадената архитектура, трябва да следва. Архитектурата дефинира списъка от инструкции, които процесора трябва или може да изпълнява и какво трябва да прави всяка една от тях. Това улеснява разработката на софтуер и подмяната на хардуер, защото софтуер, написан за един процесор, ще работи и на друг процесор със същата архитектура.

% QEMU & Virtio
Виртуална машина е софтуер, който емулира реален компютър. Най-често се използват за емулацията на несъществуващ хардуер или за изпълнението на няколко отделни системи върху един и същи хардуер. Най-големият им минус е загубата на скорост, която идва от емулацията. Разработени са методи за намаляването на този проблем, един от които е симулирането на специални Virtio устройства вместо целият хардуер.

% encryption
% цел на проекта
След разглеждане на горепосочените описания, целта на проекта става ясна: приложението трябва да работи във витруална машина, създадена от QEMU, без да използва операционна система.

% TODO - спомени Virtio в целта
% TODO - какво прави приложението?
