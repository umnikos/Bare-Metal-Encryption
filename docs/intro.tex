% introduction (1-2 pages)
\chapter*{Увод}
\addcontentsline{toc}{chapter}{Увод}

% programming languages
% Компютрите в основата си са прости машини, които четат инструкции и ги изпълняват една след друга. Инструкциите, които един компютър разбира, са много прости и за да се направи нещо по-сложно е нужен голям брой от тях. Това прави разработката на софтуер изцяло на асемблер бавен, сложен и труден процес. Именно за това са били създадени първите езици от високо ниво ({\tt COBOL}, {\tt Fortran}, {\tt C} и др.), които правят писането на код по-лесно, като автоматизират частично или напълно части от процеса на програмиране и правят кода по-четим. Цената на това удобство е загуба на контрол върху хардуера и по-бавен код. Колкото по-високо е нивото на езика, толкова повече се засилва този ефект.

% operating systems
За изпълнението на няколко програми на един компютър е нужна още една програма, която да ги координира (scheduler). При настолните компютри тази функция я изпълнява операционната система.
Операционна система е програма, която се грижи за набор от други изпълняващи се програми (процеси), като ги снабдява с опростен (но ограничен) достъп до хардуерни функции и дели времето на процесора между тях. Операционните системи обикновено са доста големи, защото трябва да поддържат голям набор от функции, на които най-разнообразни приложения да разчитат.

% QEMU
Един от проблемите при използването единствено на операционна система за хостване на сървър е, че няма изолация между отделните приложения (обща файлова система, общи библиотеки, обща операционна система и др.). Това прави поддръжката на такъв сървър трудна, но е още по-голям проблем при облачния хостинг, където различните приложения принадлежат на различни собственици и може да се злоупотреби с липсата на изолация.
Съществуват две решения на този проблем - контейнеризация и виртуализация.

% containers
Контейнеризацията е процес, в който едно приложение се пакетира заедно с всичките му нужни библиотеки, което прави преместването на приложението от една система на друга лесно и бързо. Това е достатъчно добра изолация, за да улесни поддръжката на обикновен сървър, но не и за облачният хостинг, където контейнерите допълнително се изолират, за да е достатъчно сигурна системата.

% virtualization
Хипервайзор е програма, която позволява изпълнението на няколко операционни системи върху едно устройство едновременно, като всяка операционна система се намира във виртуална машина, наподобяваща истинския хардуер. За разлика от операционна система, виртуалната машина не скрива хардуерът, а само го поделя между отделните приложения (като това става чрез специални хардуерни функции на процесора). Изолацията между отделните операционни системи е пълна и всяко приложение може да получи собствена операционна система, която да съдържа само това, от което се нуждае приложението.

% bare metal
Едно от нещата, които контейнерите не позволяват, а виртуализацията позволява, е да се използват bare-metal приложения (на мястото на операционна система).
% Не винаги е добра идея да се използва операционна система.
Всяко приложение, което не ползва операционна система, се нарича bare-metal (``гол-метал'') приложение. Това са най-вече приложения, които вървят върху маломощен хардуер (микроконтролер) или се нуждаят от максимално бързодействие. Поради липсата на операционна система тези приложения трябва сами да извършват някои от нейните функции и се пишат по-трудно и по-бавно.

% x86
% Процесорна архитектура е колекция от правила, които всеки процесор, имплементиращ дадената архитектура, трябва да следва. Архитектурата дефинира списъка от инструкции, които процесора трябва или може да изпълнява и какво трябва да прави всяка една от тях. Това улеснява разработката на софтуер и подмяната на хардуер, защото софтуер, написан за един процесор, ще работи и на друг процесор със същата архитектура.

% encryption
% цел на проекта
След разглеждане на горепосочените описания, целта на проекта става ясна: приложението трябва да работи във виртуална машина, създадена от QEMU, без да използва операционна система. Това, което ще прави приложението, е да криптира и декриптира данни с асиметрично криптиране, получени от някакъв вход. Тази апликация има две причини да е подходяща за bare-metal:
\begin{enumerate}
  \item Скорост: Асиметричното криптиране е сравнително бавно, което го прави неподходящо за доста цели. Премахването на операционната система ще забърза поне малко имплементацията.
  \item Сигурност: Ако ключът за декриптиране се генерира и складира от едно единствено устройство, то устройството един вид се превръща във физически ключ. Понеже устройството няма операционна система, то няма как да бъде заразено с вирус или друг злонамерен софтуер и единствено може да бъде хакнато чрез дупка в сигурността на интерфейса или чрез физически достъп.
\end{enumerate}


