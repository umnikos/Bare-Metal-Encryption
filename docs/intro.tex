% introduction (1-2 pages)
\section*{Въведение}
Писането на код на ниско ниво винаги е било предизвикателно начинание, изискващо от програмиста да знае подробности за хардуера, на който ще се изпълнява кодът му. Това го прави интересно и може би дори поучително, което е достатъчно добра причина човек да се захване със задачата да напише нещо на по-ниско ниво.

В тази документация ще разкажа какво съм научил при създаването на мой проект от ниско ниво - приложение за асиметрично криптиране и декриптиране на данни, работещо без операционна система, комуникирайки с външния свят чрез серийна връзка.

В първа глава ще обясня всичко което е нужно за да бъде разбрана останалата част от документацията. Това включва общи факти за работата на един компютър и подробности за virtio интерфейсът и това какво е асиметрично криптиране. Във втора глава ще се гмурнем в математиката на RSA алгоритъма за криптиране, като също ще обясня и няколко други полезни алгоритъма използвани в кода. В трета глава ще покажа самия код, какви компромиси съм направил в процеса на имплементиране и като цяло как изглежда проекта. В четвърта глава съм описал стъпка по стъпка как се компилира и изпълнява целия проект за тези които искат да го пробват.

Надявам се този документ да бъде интересен или полезен на бъдещите програмисти и да ги улесни в направата на техни, различни проекти.
\addcontentsline{toc}{section}{Въведение}
