% conclusion (1 page)
\chapter*{Заключение}
\addcontentsline{toc}{chapter}{Заключение}
% FIXME - make it longer
Има твърде много, което може да се каже за криптирането и програмирането на ниско ниво, че да се побере изцяло в която и да е документация. Проектът би могъл да се разшири по много начини, някои от които да се добавят още драйвери (за външен източник на случайност за по-голяма сигурност, за твърд диск за дългосрочно съхранение на ключовете и др.), още функционалности (потвърждаване на изпращачът на съобщение), още оптимизации (подобрение на алгоритъма за деление на големи числа).

Според мен най-голямото постижение на проекта е, че научих много докато го правех. Един от алгоритмите, които използвах в проекта (алгоритъма за бинарно експониране), ми послужи дори по време на националната олимпиада по информатика през 2021г. Това също беше първият ми сериозен проект, който да включва код на асемблер, и за в бъдеще този опит със сигурност ще ми бъде полезен.

