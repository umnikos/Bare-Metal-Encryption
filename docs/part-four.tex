% usage manual (6-10 pages)
\section{Ръководство на потребителя}
\subsection{Сваляне на кода от интернет}
Кодът е качен в GitHub и може да се свали локално чрез {\tt git}:
\begin{lstlisting}
git clone https://github.com/umnikos/Bare-Metal-Encryption.git
cd Bare-Metal-Encryption
git submodule init
git submodule update
\end{lstlisting}

\subsection{Компилиране на документацията}
Документацията е написана на \LaTeX{}. Намира се в папка {\tt docs} и най-лесно се компилира със следните команди (за линукс):
\begin{lstlisting}
pdflatex docs.latex
bibtex docs
pdflatex docs.latex
pdflatex docs.latex
\end{lstlisting}
Ако нямате {\tt pdflatex} или {\tt bibtex} инсталирани то те се инсталират по следния начин (на Debian-базирани линукс дистрибуции):
\begin{lstlisting}
sudo apt update
sudo apt install texlive-latex-base texlive-lang-cyrillic
sudo apt install texlive-latex-extra biber
\end{lstlisting}
Ако получите грешка при компилация е възможно горните команди да я оправят.

\subsection{Компилиране на приложението}
Различни конфигурации на кода се намират в папка {\tt builds}. Пример (за линукс):
\begin{lstlisting}
cd docs/bare-metal-encryption
make main-serial
make run-serial
\end{lstlisting}
Последната команда е за изпълняване на кода чрез QEMU. QEMU се инсталира по следния начин (за Debian-базирани линукс дистрибуции):
\begin{lstlisting}
sudo apt update
sudo apt install qemu-system-x86
\end{lstlisting}

За компилация е нужно да имате инсталиран i686 C cross-compiler.\footnote{Експериментирах с {\tt zig cc -target i386-freestanding} като C компилатор, но не успях да постигна успех при свързването на изпълнимия файл.} Такъв компилатор не се предлага в {\tt apt} а трябва да се компилира от сорс. За тази цел са нужни няколко програми които се инсталират със следните команди (за Debian-базирани линукс дистрибуции):
\begin{lstlisting}
sudo apt update
sudo apt install build-essential bison flex libgmp3-dev
sudo apt install libmpc-dev libmpfr-dev texinfo
\end{lstlisting}
Сорс кодовете може да изтеглите от https://ftp.gnu.org/gnu/gcc/ и https://ftp.gnu.org/gnu/binutils/ като {\tt .tar.gz} файлове.

След като свалите кода на {\tt gcc} и {\tt binutils} следва да ги компилираме:
\begin{lstlisting}
export PREFIX="$HOME/opt/cross"
export PATH="$PREFIX/bin:$PATH"
export TARGET=i686-elf

cd ~/Downloads
tar zxf gcc-10.2.0.tar.gz
tar zxf binutils-2.36.tar.gz

mkdir build-binutils
cd build-binutils
../binutils-2.36/configure --target=$TARGET --prefix="$PREFIX" \
    --with-sysroot --disable-nls --disable-werror
make
make install

cd ..
mkdir build-gcc
cd build-gcc
../gcc-10.2.0/configure --target=$TARGET --prefix="$PREFIX" \
    --disable-nls --enable-languages=c,c++ --without-headers
make all-gcc
make all-target-libgcc
make install-gcc
make install-target-libgcc

\end{lstlisting}

{\tt TARGET} е мястото където ще се инсталира компилатора и може да бъде променено. Имайте в предвид че компилацията отнема време. Тези команди трябва да се изпълнят в един и същи терминал в този ред.

След рестартиране на терминала ще откриете, че компилатора вече не може да бъде намерен, това е защото първите две команди са временни. Проблемът се оправя като ги добавите в края на {\tt \~{}/.bashrc} файла за да се изпълняват при всяко отваряне на терминала.




