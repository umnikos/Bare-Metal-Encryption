% usage manual (6-10 pages)
\section{Ръководство на потребителя}
\subsection{Компилиране на документацията}
Документацията е написана на \LaTeX{}. Намира се в папка {\tt docs} и най-лесно се компилира със следните команди (за линукс):
\begin{lstlisting}
pdflatex docs.latex
bibtex docs
pdflatex docs.latex
pdflatex docs.latex
\end{lstlisting}

\subsection{Компилиране на приложението}
Различни конфигурации на кода се намират в папка {\tt builds}. Пример (за линукс):
\begin{lstlisting}
cd docs/bare-metal-encryption
make main-serial
make run-serial
\end{lstlisting}
Последната команда е за изпълняване на кода чрез QEMU. За компилация е нужно да имате инсталиран i686 C cross-compiler.\footnote{Ако не ви се занимава да компилирате gcc като cross-compiler може да използвате {\tt zig cc -target i386-freestanding} като C компилатор, но не успях да постигна успех при свързването на изпълнимия файл по този начин.}

