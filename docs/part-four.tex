% usage manual (6-10 pages)
\chapter{Ръководство на потребителя}
  \section{Сваляне на кода от интернет}
  Кодът на приложението и документацията е качен в GitHub и може да се свали локално чрез {\tt git}:
  \begin{lstlisting}
  git clone --recurse-submodules \
      https://github.com/umnikos/Bare-Metal-Encryption.git
  \end{lstlisting}

  \section{Компилиране на документацията}
  Документацията е написана на \LaTeX{}. Намира се в папка {\tt docs} и най-лесно се компилира със следните команди (за Linux):
  % FIXME - minted changed the commands and requirements!
  \begin{lstlisting}
  pdflatex docs.latex
  bibtex docs
  pdflatex docs.latex
  pdflatex docs.latex
  \end{lstlisting}
  Ако на машината липсват {\tt pdflatex} или {\tt bibtex} или се получават грешки при компилация то трябва да се инсталират всички нужни \LaTeX{} пакети. Инсталацията става по следният начин (за Debian-базирани Linux дистрибуции):
  \begin{lstlisting}
  sudo apt update
  sudo apt install texlive-latex-base texlive-lang-cyrillic
  sudo apt install texlive-latex-extra biber
  \end{lstlisting}

  \section{Компилиране на приложението}
  % TODO

  \section{Начин на ползване}
  % PICTURES
  % TODO
