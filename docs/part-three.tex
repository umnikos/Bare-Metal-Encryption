% implementation (10-15 pages)
% specifics
\chapter{Имплементация} % FIXME
\section{Структура на проекта}
Структурата на проекта няколкократно беше променена в процеса на разработка, като в тази документация ще бъде описан само крайната структура.

В главната директория на проекта се съдържат няколко директории. Директория {\tt docs} съдържа единствено \LaTeX{} файловете, които са нужни за компилиране на документацията, както и други материали (изображения). Всички останали директории без директория {\tt builds} са различни части от кода, които могат да бъдат компилирани в различни комбинации. Различните комбинации се намират в директория {\tt builds}. Кратко описание на всички директории (освен {\tt docs} и {\tt builds}):
\begin{itemize}
  \item {\tt kernel} - Основни функционалности, свързани с работата върху bare-metal (инициализации при стартиране на компютъра и функции от ниско ниво като пускане и спиране на прекъсванията).
  \item {\tt virtio} - Имплементация на virtio-serial сериен драйвер.
  \item {\tt serial} - Имплементация на обикновен сериен драйвер.
  \item {\tt tiny-bignum-c} - Използваната библиотека за големи (дълги) числа.
  \item {\tt encryption} - Имплементация на RSA криптиране/декриптиране.
\end{itemize}

Различното комбиниране на кода е постигнато чрез използването на различни имена за входно-изходните операции в криптирането (което ги използва) и драйверите (които ги предоставят). Различното свързване след това става като се създаде един единствен {\tt C} файл, който да имплементира едните функции чрез другите. Пример за това е даден във фигура \ref{fig:io-file-example}.
\begin{figure}[ht]
  \centering
  \caption{Свързване на {\tt serial} драйвер като изход за {\tt encryption}}
  \begin{minted}{C}
    #include "../../encryption/prelude.h"
    #include "../../serial/serial.h"

    void write_out(const char* str) {
      serial_send(str);
    }

    void read_in(char* buf, i size) {
      serial_receive(buf, (u32)size);
    }

    void flush_out() {
      return;
    }
  \end{minted}
  \label{fig:io-file-example}
\end{figure}

\section{Реализация на минимално ядро}
  \subsection{Първи задачи на ядрото}
  Първото, което трябва да направи ядрото е да създаде стек за да може кодът да бъде организиран като функции. След това се изключват прекъсванията и се създават (и зареждат) GDT и IDT таблиците. След като е създадена IDT таблицата прекъсванията могат отново да се включат. След тези стъпки може да се извика {\tt C} функцията {\tt kernel\_main}, която се дефинира в директория {\tt builds} според това какво се иска да върши приложението.

  \subsection{Създаване на стек}
  Заделянето на памет може да се постигне в {\tt C} като се създаде статичен масив, но същият ефект е постигнат във фигура \ref{fig:stack-alloc} директно чрез асемблер. За да може стекът да е съвместим с {\tt C} трябва стойността на {\tt esp} регистъра винаги да се дели на 16\cite{sysvabi}. Това най-лесно се постига като стекът започне подравнен към 16 байта и се поддържа подравнен от всички следващи функции.

  \begin{figure}[ht]
    \centering
    \caption{Заделяне на памет за стекът}
    \begin{minted}{nasm}
      section .bss
      align 16
      stack_bottom:
          resb 16384
      stack_top:
    \end{minted}
    \label{fig:stack-alloc}
  \end{figure}

  Зареждането на стека е показано във фигура \ref{fig:stack-esp}. Адресът, който се зарежда в {\tt esp} е с едно по-голям от най-високият адрес в заделената памет. Това е защото при x86 стекът започва от високите адреси и върви надолу към ниските.

  \begin{figure}[ht]
    \centering
    \caption{Зареждане на стека в {\tt esp} регистъра}
    \begin{minted}{nasm}
      mov esp, stack_top
    \end{minted}
    \label{fig:stack-esp}
  \end{figure}

  \subsection{Изключване и включване на прекъсванията}
  Това става лесно с {\tt cli} и {\tt sti} асемблер инструкциите, които правят точно това. Изведени са като функции за удобство. Фигура \ref{fig:interrupt-disable} показва кода.

  \begin{figure}[ht]
    \centering
    \caption{Включване и изключване на прекъсванията}
    \begin{minted}{nasm}
      global disable_interrupts
      disable_interrupts:
          cli
          ret

      global enable_interrupts
      enable_interrupts:
          sti
          ret
    \end{minted}
    \label{fig:interrupt-disable}
  \end{figure}

  \subsection{Създаване на GDT таблица}
  В паметта трябва да се заделят две неща. Първото е самата таблица, където всеки ред е широк 8 байта и съдържа информация за сегмента, включително къде започва, къде свършва и от кой тип е. Фигура \ref{fig:gdt} показва GDT таблица със само един сегмент за код и един сегмент за данни, като и двата сегмента са за код с ниво на привилегия 0 (други нива не са използвани в проекта). Второто, което трябва да се задели, е 6-байтов дескриптор, оказващ размера на таблицата (2 байта) и адресът към нея (4 байта). Фигура \ref{fig:gdtr} го показва.

  \begin{figure}[ht]
    \centering
    \begin{minted}{nasm}
      gdt:
          ; null segment (0x00) (left empty)
          dw 0 ; limit 0-15
          dw 0 ; base 0-15
          db 0 ; base 16-23
          db 0 ; type
          db 0 ; limit 16-19 | flags
          db 0 ; base 24-31

          ; code segment (0x08)
          dw 0xFFFF
          dw 0
          db 0
          db 0x9A
          db 0xCF
          db 0

          ; data segment (0x10)
          dw 0xFFFF
          dw 0
          db 0
          db 0x92
          db 0xCF
          db 0
      gdt_end:
    \end{minted}
    \caption{GDT таблица в паметта}
    \label{fig:gdt}
  \end{figure}
  \begin{figure}[ht]
    \centering
    \begin{minted}{nasm}
      gdtr:
          dw gdt_end-gdt-1
          dd gdt
    \end{minted}
    \caption{Дескриптор към GDT таблица}
    \label{fig:gdtr}
  \end{figure}

  Зареждането на таблицата (което се прави за да има ефект таблицата) е показано на фигура \ref{fig:lgdt} и става чрез {\tt lgdt} инструкцията като се подава като аргумент адресът на дескриптора. След {\tt lgdt} инструкцията таблицата все още няма ефект докато не се презапишат сегментните регистри. В защитен решим всеки сегментен регистър представлява 16 бита, които индексират ред в GDT таблицата. Най-младшите 3 бита са за нивото на привилегия и дали се използва GDT или LDT таблицата (за LDT таблицата не се говори в тази документация). Останалите 13 бита са индекс в GDT (или LDT) таблицата. За сегмента за кода индексът в таблицата е 1 и следователно в регистър {\tt cs} трябва да се зареди стойността 0x08 (00000100 в бинарна бройна система). Аналогично за сегмента на данните, чиито индекс в таблицата е 2, регистрите {\tt ds}, {\tt es}, {\tt fs}, {\tt gs} и {\tt ss} трябва да получат стойността 0x10 (00010000 в бинарна бройна система). Сегментните регистри за данни лесно се променят чрез {\tt mov} инструкцията за преместване на данни, но {\tt mov} не работи за {\tt cs} регистъра. Той може да се промени само с далечен {\tt jmp}.

    \begin{figure}[ht]
      \centering
      \begin{minted}{nasm}
        section .text
        global init_gdt

        init_gdt:
            lgdt [gdtr]
            jmp 0x08:.reload_cs
          .reload_cs:
            mov ax, 0x10
            mov ds, ax
            mov es, ax
            mov fs, ax
            mov gs, ax
            mov ss, ax
            ret
      \end{minted}
      \caption{Зареждане на GDT таблицата}
      \label{fig:lgdt}
    \end{figure}

  \subsection{Създаване на IDT таблица}
  IDT таблицата се създава по подобен начин като GDT таблицата. Тя ще бъде пълна с много повече редове (до 256) и за това може да се спори, че е по-лесна за попълване през {\tt C}. Таблицата се пълни с адресите на процедури, като функциите трябва вместо с {\tt ret} да приключват с {\tt iret}. Прекъсванията между 0x20 и 0x27 ще бъдат хардуерните прекъсвания на майстор (``master'') PIC чипа и процедурите също така трябва да го уведомят при приключването на прекъсването. Прекъсванията между 0x28 и 0x2F принадлежат на чиракът (``slave'') PIC чип и трябва да го уведомят, както и да уведомят майстор PIC чипа. Последните 2 вида процедури са показани на фигура \ref{fig:irqhandlers}

  \begin{figure}[ht]
    \centering
    \begin{minted}{nasm}
      section .text
      global idt_handler1
      global idt_handler2

      idt_handler1:
          pushad
          ; code goes here
          mov al, 0x20
          out 0x20, al
          popad
          iret

      idt_handler2:
          pushad
          ; code goes here
          mov al, 0x20
          out 0xA0, al
          out 0x20, al
          popad
          iret
    \end{minted}
    \caption{Празни процедури за отговаряне на хардуерните прекъсвания}
    \label{fig:irqhandlers}
  \end{figure}

  Конфигурацията на PIC чиповете е показана във фигура \ref{fig:picconfig} ({\tt out\_byte} е функция, която опакова инструкцията {\tt out} за да може да се използва през {\tt C}). След като се конфигурират идва ред на това да се попълни IDT таблицата, да се направи дескриптор и да се зареди чрез {\tt LIDT}, което става аналогично на {\tt GDT} таблицата.

  \begin{figure}[ht]
    \centering
    \begin{minted}{C}
      out_byte(0x20, 0x11); // start initialization sequence
      out_byte(0xA0, 0x11);

      out_byte(0x21, 0x20); // master PIC offset 0x20
      out_byte(0xA1, 0x28); // slave PIC offset 0x28 (0x20+8)
      out_byte(0x21, 0x04); // tell master to cascade to slave through IRQ2
      out_byte(0xA1, 0x02); // tell slave its cascade identity
      out_byte(0x21, 0x01); // 8086 mode
      out_byte(0xA1, 0x01);
      out_byte(0x21, 0); // set masks to 0 (all interrupts enabled)
      out_byte(0xA1, 0);
    \end{minted}
    \caption{Конфигурация на PIC чиповете}
    \label{fig:picconfig}
  \end{figure}

\section{Реализация на virtio-serial драйвер}
  \subsection{Търсене на Virtio устройство по PCI}
  Всички Virtio устройства имат един и същи номер на производител (0x1AF4), докато номерът на устройството (по-точно за старите virtio устройства) е между 0x1000 и 0x103F. Фигура \ref{fig:finding-virtio} показва прост алгоритъм за итериране през всички PCI устройства (идентифицирани чрез двойката числа {\tt bus} и {\tt device}) и проверяване дали е Virtio устройство. Това не е най-бързият метод, но е най-простият и работи достатъчно ефективно.

  След като бъде намерено устройството по PCI е добра идея да бъде запазена двойката числа {\tt bus} и {\tt device} за да не трябва устройството да бъде търсено отново. Чрез тези две числа може да бъде взет портът му за комуникация през {\tt in/out} инструкциите (числото {\tt iobase} във фигурата) с Virtio устройството, както и номерът на хардуерното прекъсване, което предизвиква.

  \begin{figure}[ht]
    \centering
    \begin{minted}{C}
      void pci_find_virtio(struct virtio_device* result) {
        for(u32 bus=0; bus<256; bus++) {
          for(u32 device=0; device<32; device++) {
            u16 deviceid = pci_read_deviceid(bus, device);
            if (deviceid >= 0x1000 &&
                deviceid <= 0x103F && // search only for old virtio devices
                pci_read_vendor(bus,device) == 0x1AF4) {
              result->bus = bus;
              result->device = device;
              result->iobase = pci_read_bar(bus, device, 0) & 0xFFFC;
              return;
            }
          }
        }
        crash("NO VIRTIO DEVICE FOUND");
      }
    \end{minted}
    \caption{Търсене на Virtio устройство по PCI}
    \label{fig:finding-virtio}
  \end{figure}

  \subsection{Инициализиране на Virtio устройство}
  % TODO

  \subsection{Договаряне с Virtio устройство}
  % TODO

  \subsection{Създаване на virtqueue опашките}
  % TODO

  \subsection{Конфигуриране на Virtio устройство}
  % TODO

\section{Реализация на RSA}
  \subsection{Източник на случайност}
  За RSA е най-добре да се използва източник на истинска случайност, но не може да се разчита на всеки компютър да има такъв. Асемблерската инструкция {\tt rdrand} за генериране на истински случайни числа за пръв път се появява в x86 архитектурата през 2012 година и процесори, създадени преди тази година, няма начин да имат тази инструкция. Външен хардуер като източник на случайност също е опция, но това изисква имплементирането на още един драйвер за нещо, което не е основен приоритет на проекта.

  Но все пак трябва да се имплементира някакъв източник на случайност. За целта беше избран ``Xorshift'' - изключително лесен за имплементира алгоритъм за псевдослучайни числа. Имплементацията е показана на фигура \ref{fig:xorshift}.

  Всички алгоритми за псевдослучайни числа изискват да бъде подадено първоначално случайно число (``seed''), чрез което да бъде започнато генерирането на всички останали числа. Това число може да бъде генерирано чрез точният час, което има ефектът да генерира случайно число базирано на това точно кога е бил включен компютъра от потребителя. Но дори без достъп до точен час може да се използва този метод по начинът, показан във фигура \ref{fig:seed-generation}. Часът се взима от CMOS чипът. Това е чип в компютъра със собствено захранване, който работи дори когато компютъра е изключен и поддържа точен час. Фигура \ref{fig:cmos-time} показва как да се вземат секундите от този чип (тъй като реално само те са нужни).

  \begin{figure}[ht]
    \centering
    \caption{Имплементация на Xorshift алгоритъм за случайни числа}
    \begin{minted}{C}
      int seed;

      int rng() {
        seed ^= seed << 13;
        seed ^= seed >> 17;
        seed ^= seed << 5;
        return seed;
      }
    \end{minted}
    \label{fig:xorshift}
  \end{figure}

  \begin{figure}[ht]
    \centering
    \caption{Генериране на първоначално число за Xorshift}
    \begin{minted}{C}
      void init_rng() {
        int timestamp = get_time();
        int millis = 0; // not really milliseconds
        while (get_time() == timestamp) ++millis;
        seed = millis * ((timestamp << 17) + 17);
      }
    \end{minted}
    \label{fig:seed-generation}
  \end{figure}

  \begin{figure}[ht]
    \centering
    \begin{minted}{C}
      int get_time() {
        out_byte(0x70, 0x80);
        int seconds = in_byte(0x71);
        return seconds;
      }
    \end{minted}
    \caption{Взимане на времето от CMOS чипа}
    \label{fig:cmos-time}
  \end{figure}

  \subsection{Имплементация на разширеният алгоритъм на Евклид без големи отрицателни числа}
  От {\tt tiny-bignum-c} библиотеката не се поддържат отрицателни числа. Това не е проблем за RSA алгоритъма (както е описан в точка \ref{rsa}), тъй като всички използвани числа са положителни, освен при изчисляването на стойността на $k$.

  Тъждеството на Безу (уравнение \ref{rsa-bezout}) не гарантира, че $x$ и $y$ няма да са отрицателни, като лесно се вижда, че ако $\gcd(a,b) < a < b$ то едно от числата $x$ и $y$ трябва да бъде отрицателно. За щастие единственото число, което е нужно, е $x$, и съществува трик чрез който да се заобиколи проблемът с липсата на големи отрицателни числа.

  Разширеният алгоритъм на Евклид намира $x$ и $y$ така че $|x|\leq|b/\gcd(a,b)|$ и $|y|\leq|a/\gcd(a,b)|$ (където $|x|$ означава абсолютната стойност на $x$). Равенство може да се получи само ако $\gcd(a,b)=a$ или $\gcd(a,b)=b$, но понеже в този конкретен случай $\gcd(a,b)=1$ и $a \neq 1 \neq b$ следва че $|x|<|b|$, и тъй като $b>0$ е гарантирано $b-x$ да e положително цяло число.
  Още повече, понеже $b$ всъщност е експонентът за криптиране, $e$, който е достатъчно малък за да се побере в обикновена 32-битова променлива, то $x$ също може да се побере в обикновена 32-битова променлива. Следователно {\tt int} дата типът може да се използва при изчисляването на $x$ и после, в зависимост от знака, да се добави или извади от $e$.

