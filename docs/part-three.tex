% implementation (10-15 pages)
\section{Реализация}
\subsection{Реализация на ядрото}
В първите 8KB на крайният изпълним файл трябва да се намира вълшебното число 0x1BADB002 което го маркира като ядро. Веднага след него следват флагове и сума за проверка.
\begin{lstlisting}
section .multiboot
align 4
    dd MAGIC_NUMBER
    dd FLAGS
    dd CHECKSUM
\end{lstlisting}                                                                                                                                 Адресът на вълшебното число трябва да е подравнено към 4 байта и за това използваме {\tt align 4}

Входът на програмата е {\tt \_start}, като това което прави е да създаде стек, да изключи прекъсванията, да създаде GDT и IDT таблици, да включи обратно прекъсванията и да извика {\tt C} функцията {\tt kernel\_main}.
\begin{lstlisting}
section .text

global _start
extern disable_interrupts
extern enable_interrupts
extern kernel_main
extern init_gdt
extern init_idt
_start:
    mov esp, stack_top
    call disable_interrupts
    call init_gdt
    call init_idt
    call enable_interrupts
    call kernel_main
global halt
halt:
    call disable_interrupts
  .loop:
    hlt
    jmp .loop
\end{lstlisting}
Функцията {\tt halt} е заположена веднага след {\tt \_start} в случай че {\tt kernel\_main} приключи работа автоматично да се извика {\tt halt} и да спре работата на компютъра.

\subsubsection{Създаване на стек}
Стекът представлява празно парче памет с начало и край и размер. {\tt stack\_top} е началото му (защото стекът върви отгоре надолу). System V ABI\footnote{ABI (Application Binary Interface) е нещо подобно на API но на ниско ниво и специфира неща от сорта на в какъв ред се слагат аргументите на функция върху стека.} спецификацията казва че при извикване на функция стекът трябва да е подравнен към 16 байта и за това го стартираме подравнен с {\tt align 16}.
\begin{lstlisting}
section .bss
align 16
stack_bottom:
    resb 16384
stack_top:
\end{lstlisting}

\subsubsection{Изключване и включване на прекъсванията}
Изключването и включването на прекъсвания става чрез {\tt cli} и {\tt sti} инструкциите, които съм направил на функции за лесно ползване от {\tt C}.
\begin{lstlisting}
global disable_interrupts
disable_interrupts:
    cli
    ret
global enable_interrupts
enable_interrupts:
    sti
    ret
\end{lstlisting}

\subsubsection{Създаване на GDT таблица}
За да се инициализира GDT таблицата тя първо трябва да бъде създадена в паметта:
\begin{lstlisting}
section .data
gdtr:
    dw gdt_end-gdt-1
    dd gdt
gdt:
    ; null segment (0x00) (has to be empty)
    dw 0 ; limit 0-15
    dw 0 ; base 0-15
    db 0 ; base 16-23
    db 0 ; type
    db 0 ; limit 16-19 | flags
    db 0 ; base 24-31

    ; code segment (0x08)
    dw 0xFFFF
    dw 0
    db 0
    db 0x9A
    db 0xCF
    db 0

    ; data segment (0x10)
    dw 0xFFFF
    dw 0
    db 0
    db 0x92
    db 0xCF
    db 0
gdt_end:
\end{lstlisting}
Първия ред в таблицата е задължителен и е просто нули. Вторият създава сегментът за кода, който се простира върху всички 4GB от паметта. Третият създава сегментът за данните, който също се простира върху всички 4GB от паметта и се препокрива с кодът. По този начин всеки адрес в паметта може да бъде прочетен, записан и изпълнен.

Зареждането на тази таблица става с {\tt lgdt} инструкцията. След като бъде заредена сегментните регистри трябва да бъдат преинициализирани за да се активират промените:
\begin{lstlisting}
section .text
global init_gdt
init_gdt:
    lgdt [gdtr]
    jmp 0x08:.reload_cs
  .reload_cs:
    mov ax, 0x10
    mov ds, ax
    mov es, ax
    mov fs, ax
    mov gs, ax
    mov ss, ax
    ret
\end{lstlisting}

\subsubsection{Създаване на IDT таблица}
Създаването на IDT таблица е малко по-сложно от създаването на GDT таблица защото има много повече редове, които трябва да се попълнят. Не е проблем ако са попълнени повече редове отколкото е нужно, затова съм попълнил 256 от тях за всеки случай. Всеки ред заема 8 байта, следователно трябва да резервираме общо $8\times 256$ байта за цялата таблица:
\begin{lstlisting}
section .data
idtr:
    dw idt_end-idt-1
    dd idt

section .bss
global idt
idt:
    resb 8*256
idt_end:
\end{lstlisting}
За да се инициализира таблицата трябва да се попълни и зареди, но попълването е леко сложно и за улеснение съм го написал на {\tt C}:
\begin{lstlisting}
extern fill_idt
global init_idt
init_idt:
    call fill_idt
    lidt [idtr]
\end{lstlisting}
Ето я и самата функция:
\begin{lstlisting}[language=C]
struct idt_entry {
  u16 offset_1; // offset bits 0..15
  u16 selector; // code segment selector in GDT
  u8 zero; // full of 0s for legacy reasons
  u8 type_attr; // settings
  u16 offset_2; // offset bits 0..16
} __attribute__((packed));

extern struct idt_entry idt[];
extern void idt_handler0();
extern void idt_handler1();
extern void idt_handler2();
extern void enable_interrupts();
extern void disable_interrupts();

void fill_idt() {
  out_byte(0x20, 0x11);
  out_byte(0xA0, 0x11);

  out_byte(0x21, 0x20);
  out_byte(0xA1, 0x28);
  out_byte(0x21, 0x04);
  out_byte(0xA1, 0x02);
  out_byte(0x21, 0x01);
  out_byte(0xA1, 0x01);
  out_byte(0x21, 0);
  out_byte(0xA1, 0);

  for (u32 i=0; i<256; i++) {
    struct idt_entry ih;
    if (i >= 0x20 && i < 0x28) {
      ih.offset_1 = (u32)idt_handler1 & 0xFFFF;
      ih.offset_2 = ((u32)idt_handler1 & 0xFFFF0000) >> 16;
    } else if (i >= 0x28 && i < 0x30) {
      ih.offset_1 = (u32)idt_handler2 & 0xFFFF;
      ih.offset_2 = ((u32)idt_handler2 & 0xFFFF0000) >> 16;
    } else {
      ih.offset_1 = (u32)idt_handler0 & 0xFFFF;
      ih.offset_2 = ((u32)idt_handler0 & 0xFFFF0000) >> 16;
    }
    ih.zero = 0;
    ih.selector = 0x08;
    ih.type_attr = 0x8E;
    idt[i] = ih;
  }
}
\end{lstlisting}
Първото което прави е да пренасочи PIC (виж \ref{pic}) контролерите да приемат прекъсвания между адреси 0x20 и 0x29 (защото предишните адреси са заети от прекъсванията на самия процесор). След това таблицата се пълни с три различни функции:
\begin{lstlisting}
section .text
global idt_handler0
global idt_handler1
global idt_handler2
extern crash
idt_handler0:
    push 0x87654321
    call crash
    iret
idt_handler1:
    pushad
    mov al, 0x20
    out 0x20, al
    popad
    iret
idt_handler2:
    pushad
    mov al, 0x20
    out 0xA0, al
    out 0x20, al
    popad
    iret
\end{lstlisting}
{\tt idt\_handler1} и {\tt idt\_handler2} при извикването им сигнализират на PIC контролерите че прекъсването е приключило. Без да се направи това прекъсването ще продължи вечно и процесора никога няма да получи следващо прекъсване. {\tt idt\_handler0} е за останалите прекъсвания които не очакваме и ако дойдат спира процесора.

\subsubsection{Задачите на {\tt kernel\_main}} % FIXME - NOT TRUE NOW
{\tt kernel\_main} единствено извиква {\tt virtio\_init} което търси и намира virtio-serial устройството и го инициализира и след това да извика {\tt hello\_world} което използва това virtio устройство за да изпрати ``Hello, World!''.
\begin{lstlisting}[language=C]
void kernel_main() {
  struct virtio_device virtio;
  virtio_init(&virtio);
  hello_world(&virtio);
}
\end{lstlisting}

\subsection{Реализация на virtio-serial драйвер}
В тази част от документацията доста често ще споменавам официалната virtio документация \parencite{virtiodocs}, която може да намерите приложена към този документ в електронен формат или онлайн.

Приключението започва във функцията {\tt virtio\_init}, която се извиква от функцията {\tt kernel\_main} и има за дел да намери и инициализира virtio-serial устройството. Преди да покажа кода на функцията ще покажа нейния тип:
\begin{lstlisting}[language=C]
struct virtio_device {
  u32 bus;
  u32 device;
  u16 iobase;
  struct virtq queues[2];
};

void virtio_init(struct virtio_device* virtio);
\end{lstlisting}
Поради липсата на {\tt malloc}, {\tt virtio\_init} приема пойнтер към празна {\tt virtio\_device} структура, която да бъде напълнена.

{\tt bus} и {\tt device} са резултатите от търсенето по PCI. Те са достатъчни че да се намери {\tt iobase} но съм го включил в структурата за улеснение. {\tt struct virtq queues[2]} са двете virtio опашки, структурата за които е описана във virtio документацията\parencite{virtiodocs} и представлява следното:
\begin{lstlisting}[language=C]
struct virtq_desc {
  uint64_t addr;
  uint32_t len;
  uint16_t flags;
  uint16_t next; // for descriptor chaining
};

// avaiable header
// only written by driver and read by the device
struct virtq_avail {
  uint16_t flags;
  uint16_t idx; // index of next free slot in ring
  uint16_t ring[]; // list of frames.
                   // they just hold an index to a descriptor
};

struct virtq_used_elem {
  uint32_t id; // index of descriptor.
               // uint32_t for padding reasons
  uint32_t len; // total number of bytes written to buffer
};

// used header
// only written by device and read by the driver
struct virtq_used {
  uint16_t flags;
  uint16_t idx; // index of next free slot in ring
  struct virtq_used_elem ring[]; // list of frames
};

struct virtq {
  struct virtq_desc *desc;
  struct virtq_avail *avail;
  struct virtq_used *used;
  uint16_t qsize; // number of buffers
                  // always a power of 2
};
\end{lstlisting}
Структурата е обяснена в част \ref{virtqueue} от тази документация.

Вече мога да обясня кодът на {\tt virtio\_init}:
\begin{lstlisting}[language=C]
#define VIRTIO_ACKNOWLEDGE 1
#define VIRTIO_DRIVER 2
#define VIRTIO_FEATURES_OK 8
#define VIRTIO_DRIVER_OK 4
#define VIRTIO_FAILED 128
#define VIRTIO_DEVICE_NEEDS_RESET 64

void virtio_init(struct virtio_device* res) {
  pci_find_virtio(res);

  virtio_for_irq = res;
  u8 irq = pci_read_irq(res->bus, res->device);
  set_irq(0x20+irq);

  u16 iobase = res->iobase;
  u8 status = VIRTIO_ACKNOWLEDGE;
  out_byte(iobase+0x12, status);

  status |= VIRTIO_DRIVER;
  out_byte(iobase+0x12, status);

  u32 supported_features = in_dword(iobase+0x00);
  virtio_negotiate(&supported_features);
  out_dword(iobase+0x04, supported_features);

  virtio_queues(res);

  status |= VIRTIO_DRIVER_OK;
  out_byte(iobase+0x12, status);
}
\end{lstlisting}
Първото което правим е да открием устройството като извикаме {\tt pci\_find\_virtio}. Функцията попълва {\tt bus}, {\tt device} и {\tt iobase} полетата от {\tt virtio\_device} структурата.

След това инициализираме прекъсванията като прочетем кое прекъсване отговаря на virtio устройството и нагласяме функцията която отговаря за прекъсването да го гледа чрез {\tt set\_irq}. Функцията също ще се нуждае от {\tt virtio\_device} структурата, но понеже няма как да и я подадем като параметър я съхраняваме като глобална променлива.

Вече може да започне самото инициализиране на устройството както е обяснено в глава \ref{virtioinit} като първата стъпка можем да я пропуснем защото устройството започва рестартирано.

Всички {\tt pci\_read\_*} команди са имплементирани чрез {\tt pci\_read\_config}, което приема bus и device номер както и изместване за да изберем коя част от PCI header-ът да бъде прочетена. Повече информация може да намерите онлайн\parencite{pciheader}, ето дефинициите на всички подобни функции в кода:
\begin{lstlisting}[language=C]
#define config_address  0x0CF8
#define config_data     0x0CFC
u16 pci_read_config(u32 bus, u32 device, u32 func, u32 offset) {
  u32 address = (u32)((bus << 16) | (device << 11) |
                     (func << 8) | (offset & 0xfc) |
                     ((u32)0x80000000));
  out_dword(config_address, address);
  u16 tmp = (u16)((in_dword(config_data) >>
                    ((offset & 2) * 8)) & 0xffff);
  return tmp;
}

u16 pci_read_headertype(u32 bus, u32 device) {
  return pci_read_config(bus,device,0,0x0E) & 0x00FF;
}

u16 pci_read_vendor(u32 bus, u32 device) {
  return pci_read_config(bus,device,0,0);
}

u16 pci_read_deviceid(u32 bus, u32 device) {
  return pci_read_config(bus,device,0,2);
}

u16 pci_read_subsystem(u32 bus, u32 device) {
  return pci_read_config(bus,device,0,0x2E);
}

u16 pci_read_irq(u32 bus, u32 device) {
  return pci_read_config(bus,device,0,0x3C) & 0x00FF;
}

u32 pci_read_bar(u32 bus, u32 device, u32 number) {
  u16 bottom = pci_read_config(bus,device,0, 0x10 + number*4);
  u16 top = pci_read_config(bus,device,0, 0x12 + number*4);
  return (top<<16)+bottom;
}
\end{lstlisting}

\subsubsection{Търсене на virtio устройство по PCI}
Търсенето става чрез итерирането през всички възможни комбинации на bus номер и device номер. При неуспешно намиране на virtio устройство просто спираме компютъра понеже проектът изцяло зависи от това да намерим virtio-serial устройството за входно-изходни операции.
\begin{lstlisting}[language=C]
void pci_find_virtio(struct virtio_device* result) {
  for(u32 bus=0; bus<256; bus++) {
    for(u32 device=0; device<32; device++) {
      u16 deviceid = pci_read_deviceid(bus, device);
      if (deviceid >= 0x1000 &&
          deviceid <= 0x103F && // only old virtio devices
          pci_read_vendor(bus,device) == 0x1AF4) {
        result->bus = bus;
        result->device = device;
        result->iobase = pci_read_bar(bus, device, 0) & 0xFFFC;
        return;
      }
    }
  }
  // hang when nothing is found
  halt();
  while (1);
}
\end{lstlisting}
{\tt iobase} взимаме от първия BAR адрес. Всеки BAR адрес може да съдържа или адрес в паметта или IO порт адрес. Най-младшият бит е 1 когато е IO адрес, а битът директно след него е запазен за бъдещи цели и не знаем какъв е. Въпреки че BAR адресът е 32 битова стойност, един IO порт е 16 бита и следователно само първите 16 бита от адреса ни трябват.

Един от проблемите на този код е че не проверява дали типът на устройството е virtio-serial устройство. Лесно се добавя тази проверка но тъй като никъде във virtio документацията не намерих ``virtio-serial'' устройство и неговия номер не добавих тази проверка. Почти съм сигурен че е номер 3 (console device).
Втори проблем е че винаги избира първото virtio устройство което намери. Това не е фатално стига да имаме само едно virtio устройство свързано.

\subsubsection{Договаряне с virtio устройство}
Договарянето е по-лесно отколкото звучи. Драйверът има право само за изключва функции, които не поддържа. Значението на всеки бит е специфично за всеки тип устройство.
\begin{lstlisting}[language=C]
void virtio_negotiate(u32* supported_features) {
  *supported_features &= 0x00000000;
}
\end{lstlisting}

\subsubsection{Създаване на virtqueue опашките}
\begin{lstlisting}[language=C]
void virtio_queues(struct virtio_device* virtio) {
  u16 iobase = virtio->iobase;
  for (u32 q_addr=0; q_addr<2; q_addr++) {
    // get queue size
    out_word(iobase+0x0E,q_addr);
    u16 q_size = in_word(iobase+0x0C);

    // allocate queue
    // formula given by section 2.4.2
    u32 sizeofBuffers = (sizeof(struct virtq_desc)*q_size);
    u32 sizeofQueueAvailable = (sizeof(u16)*(3+q_size));
    u32 sizeofQueueUsed = (sizeof(u16)*3 +
                    sizeof(struct virtq_used_elem)*q_size);
#define page_count(bytes) ((bytes+0x0FFF)>>12)
    u32 firstPageCount = page_count(sizeofBuffers +
                                    sizeofQueueAvailable);
    u32 secondPageCount = page_count(sizeofQueueUsed);
    u32 queuePageCount = firstPageCount+secondPageCount;
    u8* buf = gimme_memory(queuePageCount);
    u32 bufPage = (u32)buf >> 12;

    struct virtq* vq = &virtio->queues[q_addr];
    vq->qsize = q_size;
    vq->desc = (struct virtq_desc*)buf;
    vq->avail = (struct virtq_avail*)(buf+sizeofBuffers);
    vq->used = (struct virtq_used*)(buf+(firstPageCount<<12));

    vq->avail->idx = 0;
    vq->avail->flags = 0; // 1 if we don't want interrupts
    vq->used->idx = 0;
    vq->used->flags = 0;

    if (q_addr == 0) {
      static char input_buff[1025];
      virtq_insert(virtio, 0, input_buff, 1023, 2);
    }

    out_word(iobase+0x0E,q_addr);
    out_dword(iobase+0x08,bufPage);
  }
}
\end{lstlisting}
Едно virtio-serial устройство без никакви екстри има две опашки - първата е за вход а втората за изход. Първото което правим за всяка опашка е да вземем нейната големина. След това трябва да заделим достатъчно място в паметта за това. Формулата за изчисление на нужното място е описана във virtio документацията в глава 2.4.2. {\tt buf} става пойнтер към това пространство, а {\tt bufPage} е номерът на страницата където се намира пространството. Заделянето на нужната памет става чрез {\tt gimme\_memory} което е нещо подобно на {\tt malloc} но много по-просто. Ето я имплементацията:
\begin{lstlisting}[language=C]
#define heap_size 32
static u8 heap_start[heap_size*4096]
  __attribute__((aligned(4096)));

u8* gimme_memory(u32 pages) {
  static u32 given_pages = 0;
  if (pages + given_pages > (u32)heap_size) {
    crash(0x8BADF00D);
  }
  u8* res = heap_start + (given_pages<<12);
  given_pages += pages;
  return res;
}
\end{lstlisting}

След това караме трите пойнтера от {\tt virtq} структурата да сочат към правилните места в буфера и инициализираме стойностите в трите подструктури.

Ако инициализираме първата опашка искаме тя да започне с празен буфер за пълнене на входни данни. Това става чрез {\tt virtq\_insert} което ще обясня след малко.

Последното което трябва да направим за всяка опашка е да кажем на устройството къде се намира като подадем номерът на първата страница където се намира опашката. Размерът на една страница в нашият случай е 4096 байта.

{\tt virtq\_insert} слага буфер в посочена опашка. Ето го и кодът:
\begin{lstlisting}[language=C]
void virtq_insert(struct virtio_device* virtio,
                  u32 queue_num,
                  char const* buf, u32 len,
                  u8 flags) {
  u16 iobase = virtio->iobase;
  struct virtq* queue = &virtio->queues[queue_num];

  // find next free buffer slot
  // this doesn't do actual searching
  static u16 buf_index = 0;
  buf_index++;
  // set buffer slot's address to message string
  queue->desc[buf_index].addr = (u32)buf;
  queue->desc[buf_index].len = len;
  queue->desc[buf_index].flags = flags;

  // add it in the available ring
  u16 index = (queue->avail->idx) % (queue->qsize);
  queue->avail->ring[index] = buf_index;

  mem_barrier;

  queue->avail->idx++;

  mem_barrier;

  // notify the device
  if (driver_ok) {
      out_word(iobase+0x10,queue_num);
  }

}
\end{lstlisting}
Следваме стъпките описани в глава 3.2.1 от virtio документацията; Стъпка 1 е да сложим буферът в свободен дескриптор. Попринцип първо трябва да намерим празен дескриптор но можем и просто да броим колко дескриптори сме използвали ако никога няма да използваме повече от {\tt q\_size} буфери. Стъпка 2 е да сложим индексът на този дескриптор в списъка със свободни дескриптори. Стъпка 3 можем да я пропуснем. Стъпка 4 е да изпълним бариера на паметта, която е дефинирана по следния начин:
\begin{lstlisting}[language=C]
#define mem_barrier __sync_synchronize()
\end{lstlisting}
Това е интрукция, която казва на компилатора и на процесора да не разместват инструкции свързани с променянето на паметта между бариерата. Това го правим защото ни е важно в какъв ред ще се изпълнят следните инструкции.

Стъпка 5 е да увеличим {\tt idx}. Стъпка 6 е още една бариера. Стъпка 7 е да изпратим съобщение на устройството че сме сложили нов дескриптор в опашката но само ако вече сме приключили с инициализирането на устройството.

% \subsubsection{Отговаряне на прекъсвания}
% TODO - set_irq, virtio_handler_prelude, virtio interrupt handler

\subsubsection{``Hello, World!''}
{\tt hello\_world} е последната функция която се извиква и има за цел да изпрати ``Hello, World!''. Това се постига като просто се запише във втората опашка. Мисля че кодът сам говори за себе си:
\begin{lstlisting}[language=C]
void hello_world(struct virtio_device* virtio) {
  const u32 msglen = 16;
  static const char msg[] = "Hello, World!\n\0\0\0\0\0";

  virtq_insert(virtio, 1, msg, msglen, 0);

  while (1);
}
\end{lstlisting}

\subsection{Реализация на RSA}
\subsubsection{Източник на случайност}
За RSA е най-добре да се използва източник на истинска случайност. Проблемът е че не може да се разчита че всеки компютър има такъв. Асемблер инструкцията {\tt rdrand} първо се появява през 2012 година и компютри преди тази дата е гарантирано да не я поддържат (това включва и моят личен компютър). QEMU поддържа инструкцията но все пак реших да не я ползвам. QEMU също позволява симулирането на външен хардуерен източник на случайност, но това изисква да се напише драйвер специално за това устройство което според мен е извън рамките на проекта.

Но все пак трябваше да избера \textit{някакъв} източник на случайност за да направя и демонстрирам самото криптиране. Алгоритъма за псевдослучайни числа който избрах е xorshift защото е изключително лесен за имплементиране. Всички псевдослучайни алгоритми се нуждаят от първоначално число от което да започнат и за да го генерирам използвах хардуерния часовник който всички компютри имат (намиращ се върху CMOS чипът). Процедурата е следната:
\begin{lstlisting}[language=C]
i seed;

void init_rng() {
  i timestamp = get_time();
  i millis = 0;
  while (get_time() == timestamp) ++millis;
  seed = millis * timestamp;
\end{lstlisting}
Променливата {\tt millis} е за да вкара повече случайност базирана на точния момент в който е извикана функцията от потребителя и дори скоростта на процесора. Въпреки името тя не е направена да отмерва точен брой милисекунди.

\subsubsection{Проблеми с {\tt tiny-bignum-c}}
{\tt tiny-bignum-c} има два основни проблема - имплементацията е бавна и числата не могат да бъдат отрицателни.

Проблемът с бавната имплементация се опитах да го разреша като имплементирах наново двете най-важни операции - умножение и експониране. Въпреки големият успех в тази насока, генерирането на ключове и декриптирането все още са доста бавни. Може би щеше да е по-мъдро да потърся друга, по-добре направена библиотека.

Фактът че големите числа не могат да са отрицателни беше проблем единствено при имплементирането на разширеният евклидов алгоритъм, който не гарантира, че двете допълнителни числа които изкарва няма да са отрицателни. За щастие единственото число което ми е нужно от алгоритъма също е карантирано да е малко (виж \ref{rsaalgo} за повече детайли) и за това може да се побере в нормално 32 битово число.


