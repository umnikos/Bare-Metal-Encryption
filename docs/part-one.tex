% research (8-10 pages)
% nothing concrete about the implementation
\chapter{Проучване} % FIXME
\hfill
\section{Фунцкии на операционна система}
Въпреки, че заданието изисква приложението да не използва операционна система, то, като bare-metal приложение, трябва да върши някои от ролите на операционна система и за това е добра идея да бъдат разгледани без съсредоточаване в детайли.
  \subsection{Драйвери}
  Драйвер представлява програма, която се грижи за комуникацията с определено устройство и е специфична за това устройство. Целта на драйвера е да скрие всички конкретики, касаещи се за устройството (производител, версия, специфичности и други) с цел лесна замяна на хардуер с друг. В днешно време драйверите обикновено се пишат от производителя на устройството за някакъв набор от операционни системи, но всеки потребител би могъл да си напише собствен драйвер.

  Тъй като на bare-metal приложенията често им се налага да комуникират с периферни устройства, то те трябва да имплементират драйвери за тези устройства.

  \subsection{Многозадачност}
  В повечето операционни системи многозадачност се постига чрез процеси - всяка програма, която се изпълнява, се изпълнява в процес, и процесите се редуват кога ще се изпълняват и кога ще чакат другите процеси да свършат. В bare-metal приложение обикновено има само една или няколко задачи, които трябва да бъдат изпълнени, и следователно няма нужда от многозадачност.

  \subsection{Управление на паметта}
  Под ``памет'' се има предвид RAM паметта на компютъра. Устойствата за дългосрочно съхранение на информация (напр. твърд диск) се разглеждат в точка \ref{filesystems}.

  В операционните системи всеки процес има заделено парче памет с което може да прави каквото си поиска и не може да чете или записва никъде другаде. Това се прави най-вече с цел сигурност против зловредни програми и с цел една некоретно работеща програма да не повреди цялата операционна система. В bare-metal приложение поради факта, че няма да се изпълнява чужд код и поради малкия брой задачи това не е задължителна част от имплементацията, но е препоръчителна.

  \subsection{Файлова система} \label{filesystems}
  Файловите системи целят да организират дългосрочната информация като файлове, които да са разпределени в директории. Всяка файлова система се състои от голям брой сложни части и рядко е добра идея да се имплементира в bare-metal приложение.

\section{Писане на bare-metal приложение за x86 компютър}
  \subsection{Съществуващи bare-metal приложения}
  % TODO

  \subsection{QEMU/KVM}

  \subsection{Реален и защитен режим на процесора при x86}

  \subsection{Сегментация при x86}

  \subsection{Странициране при x86}

  \subsection{Стекът при x86}

  \subsection{GDT (Global Descriptor Table)}

  \subsection{IDT (Interrupt Descriptor Table)}

  \subsection{PCI (Programmable Interrupt Controller)} % Как се поддържат хардуерни прекъсвания чрез PIC

  \subsection{Стъпки при стартиране на x86 компютър}

\section{Virtio}

\section{Криптиране}
  \subsection{Какво е криптиране}

  \subsection{Симетрично криптиране}
    \subsubsection{Дифи-Хелман}

  \subsection{Асиметрично криптиране}
    % encryption *and* sender checking

